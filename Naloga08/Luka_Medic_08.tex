\documentclass[paper=a4, fontsize=12pt]{scrartcl} %,twoside

\usepackage[a4paper,pdftex]{geometry}										
\setlength{\oddsidemargin}{5mm}												
\setlength{\evensidemargin}{5mm}
\setlength{\paperheight}{29,7cm}
\setlength{\topmargin }{0cm}

%\usepackage[utf8]{inputenc}
\usepackage[slovene]{babel}
%\usepackage[T1]{fontenc}
\usepackage[protrusion=true,expansion=true]{microtype}	
\usepackage{amsmath,amsfonts,amsthm,amssymb}
\usepackage{graphicx}
%\graphicspath{ {figs2/} }
\usepackage{subcaption}
\usepackage[utf8x]{inputenc}
\usepackage{ucs}
\usepackage{sectsty}
\usepackage{pdfpages}
\usepackage{caption}
\usepackage{float}
\usepackage{url}
\usepackage{hyperref}
\usepackage{wasysym}
\usepackage[labelfont=bf]{caption}
\usepackage{makeidx}
\usepackage[nottoc,numbib]{tocbibind}
\usepackage{authoraftertitle}
\usepackage{titlecaps}
\usepackage[export]{adjustbox}
\usepackage{sidecap}
\usepackage{mathrsfs}
\usepackage{mhchem}
\usepackage{rotating}
\usepackage{wrapfig}
\usepackage{bbm}

\DeclareCaptionFormat{myformat}{\fontsize{10}{10}\selectfont#1#2#3}
\captionsetup{format=myformat}

\usepackage{dcolumn}
\newcolumntype{d}[1]{D{.}{\cdot}{#1}}
\newcolumntype{.}{D{.}{.}{-1}}
\newcolumntype{,}{D{,}{,}{2}}

\newcommand{\forcet}{\leavevmode{\parindent=1em\indent}}

\usepackage{afterpage}
\newcommand\blankpage{%
    \null
    \thispagestyle{empty}%
    \addtocounter{page}{0}%
    \newpage}

\newcommand{\HRule}[1]{\rule{\linewidth}{#1}} 	% Horizontal rule
\newcommand\id{\ensuremath{\mathbbm{1}}} 

%Definicije
\makeatletter							% Title
\def\printtitle{%						
    {\centering \@title\par}}
\makeatother									

\makeatletter							% Author
\def\printauthor{%					
    {\centering \large \@author}}			
\makeatother
 
\makeatletter	
\newcommand{\thickhline}{%
	\noalign {\ifnum 0=`}\fi \hrule height 1pt
	\futurelet \reserved@a \@xhline
}  
\makeatother
\sectionfont{\fontsize{20}{22}\selectfont}
\subsectionfont{\fontsize{16}{18}\selectfont}

\renewcommand{\d}[1]{\ensuremath{\operatorname{d}\!{#1}}}
\newcommand{\code}[1]{\texttt{#1}}

\begin{document}

\begin{flushright}
	\textbf{Luka Medic}\normalfont{, vpisna št.:} \textbf{28182014}\\
\end{flushright}
\begin{center}
\vspace*{2cm}
\large{\titlecap{\scshape Višje računske metode}}\\[0.5cm]	% Title
\huge{\textbf{8. naloga -- Schmidtov razcep in matrično produktni nastavki}} \\[0.5cm]
\normalsize \normalfont \today\\[2cm]
\end{center}

\section{Uvod}
V nalogi je predstavljen kvantni Monte Carlo, ki je preveden na klasičen problem polimerne verige. Obravnavana sta primera harmonskega in anharmonskega oscilatorja s pot-integralnim Monte Carlo algoritmom. Rezultati so primerjani z analitičnimi rešitvami in rezultati spektralne metode.

\section{Osnovno stanje Heisenbergove verige}
Imejmo antiferomagnetno ($J=-1$) Heisenbergovo verigo sode dolžine $n$:
\begin{equation}
	H = \sum_{i=1}^n \vec{s}_i \cdot \vec{s}_{i+1},
\end{equation}
s točno diagonalizacijo pa želimo poiskati osnovno stanje verige.
$H$ prestavimo z matriko velikosti $2^n \times 2^n$, ki jo konstruiramo z vsotami $n$ tenzorskih produktov Paulijevih matrik in identitet. Ker je večina elementov ničelnih, se v implementaciji splača uporabiti \code{sparse} matrike:\\
\code{>>}  \code{for k in range(n-1):}\\
\code{>>}\qquad\code{H += sparse.kron(sparse.identity(2**k),}\\
\code{>>}\qquad\qquad\qquad\qquad\code{sparse.kron(h2, sparse.identity(2**(n-k-2))))}\\
pri tem je \code{h2} matrika:
\begin{equation}
	\code{h2} = \begin{pmatrix} 
      1 & 0 & 0 & 0 \\ 
      0 & -1 & -2 & 0 \\ 
      0 & -2 & -1 & 0 \\
      0 & 0 & 0 & 1 
   \end{pmatrix}.
\end{equation}
Tako konstruiran $H$ ustreza verigi z zaprtimi robnimi pogoji. Če želimo dodati periodične robne pogoje moramo povezati prvi in zadnji spin, kar storimo z:\\
\code{>>}  \code{for s in (sx, sy, sz):}\\
\code{>>}\qquad\code{H+=np.real(sparse.kron(s,sparse.kron(sparse.identity(2**(n-2)),s)))}\\

Od tod naprej za oba primera uporabimo \code{sparse.linalg.eigsh}, da dobimo lastna stanja $H$.

Ker $H$ komutira s skupno projekcijo spina $S_z$, bi lahko problem obravnavali po sektorjih z enakimi $S_z$. Pričakujemo, da bo osnovno stanje za antiferomagnetni Heisenbergov model iz sektorja z $n_\uparrow=n_\downarrow=n/2$. Če za osnovno stanje predpostavimo, da noben koeficient iz tega sektorja ni ničeln, potem lahko predpostavko preprosto preverimo s prikazom razmerij neničelnih koeficientov izračunanega osnovnega stanja s celotnim številom stanj iz tega sektorja $\binom{n}{n/2}$. Rezultati za periodično in neperiodično verigo so prikazani na slikah ?? in ??.

%\begin{figure}[H]
%	\centering
%	\includegraphics[page=1,scale=0.5]{slika2_1}
%	\caption{Prikazane so funkcijske odvisnosti povprečnih energij od inverzne temperature $\beta$. Za harmonski oscilator velja virialni teorem: $\langle V \rangle = \langle T \rangle = \frac{\langle  H\rangle}{2}$, pri čemer je $\langle  H\rangle = \frac{1}{2}\coth(\beta/2)$.}
%	\label{fig:slika2_1}
%\end{figure}


\newpage
\section{Zaključek}


\end{document}